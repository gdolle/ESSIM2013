
\begin{abstract}

Modelling human-arm kinematics is very important in many fields. For instance, physiotherapists need to measure the amount of a person's incapability in a specific movement. This measurement may improve the efficiency of rehabilitation and also forecast its duration.

The aim of this project is to simulate the human-arm movement in the sagittal plane and compute the area of the surface generated by the hand motion.

Different strategies have been used to approach this problem. More precisely, the area can be calculated using the Monte-Carlo method or Green's theorem.

\end{abstract}

\section{Introduction}

The purpose of this project is to develop a model for the human-arm kinematics and to use it in order to calculate the area that a person can reach.

%We can consider three orthogonal planes, namely:
The motion of the human-arm can be divided into movements in three orthogonal planes, namely:

\begin{itemize}
  \item sagittal plane (S);
  \item frontal plane (P);
  \item transversal plane (C).
\end{itemize}

\begin{figure}[!ht]
  \centering
  \includegraphics[width=0.4\linewidth]{figs/arm1a}
\caption{Planes \cite{MR2345192}}
\label{fig:bodyarms}
\end{figure}

\noindent In this project, it will only be considered the sagittal plane.

The default position of the arm is open wide and orthogonal to the ground surface. The reference angles are illustrated in the figure \ref{fig:bodyarmstwo}.

\begin{figure}[!ht]
  \centering
  \includegraphics[width=0.4\linewidth]{figs/arm3aChange}
\caption{Reference angles \cite{MR2345192}}
\label{fig:bodyarmstwo}
\end{figure}

%Thus the idea is to calculate this area to give a standard measure of the typical human arm %hability.
%The problem can appear simple due to some joints constraints.
%Evaluate an area of this set to get one parameter reflecting the mobility of the arm  find a %relationship of the mobility of the joints and this area.

%Why do we need this are?  For calculation percentage of the reachable area for
%disabled people, first of all, area where arm can move, should be calculated
%numerically. For this purpose it was decided to use two numerical methods: mc
%and green.

%The parameters and their boundary conditions are listed in the table \ref{tab:parameters}. %Move
%
%%The angles between two bones linked by a joint are equal to 0.
%
%\begin{table}[!h!]
%\begin{center}
%\begin{tabular}{|c|c|c|}
%\hline
%\bf Description & \bf Variable & \bf Constraints \\\hline
%Arm length & $l_A$ &  28 cm \\\hline
%Forearm length  & $l_F$ &  28 cm \\\hline
%Palm length & $l_H$ &  18 cm  \\\hline
%Shoulder angle & $\alpha$ & $[-50^{\circ},90^{\circ}]$ \\\hline
%Elbow angle & $\beta$  & $[-10^{\circ},150^{\circ}]$  \\\hline
%Hand angle & $\gamma$ & $[-70^{\circ},80^{\circ}]$  \\\hline
%\end{tabular}
%\caption{Standard parameters}
%\label{tab:parameters}
%\end{center}
%\end{table}
% ------------------------------------------------------------------
% ------------------------------------------------------------------


\section{Model}

Depending on the arm configuration, the hand can reach different points in the plane.

Figure \ref{fig:armmodel} describes our model of the human-arm movement in the S-plane. $(l_A,l_F,l_H)$ represents the length of the arm, of the forearm and of the
hand, respectively. Moreover, $(\alpha,\beta,\gamma)$ represents the angles of the joints for the shoulder, the elbow and the wrist.

\begin{figure}[!htp]
  \centering
  \includegraphics[width=0.7\linewidth]{figs/scheme}
\caption{Arm model scheme}
\label{fig:armmodel}
\end{figure}
 
%To describe the movement of the arm, three points corresponding to each arm
%joints and to the longest fingertip are defined.
%The 2 dimension coordinates in the referential $(x,y)$ for the elbow joint point $(X_E,Y_E)$ %are,

With this referential, the expression for the elbow joint point is

\begin{align}\label{firstcoord}
X_E = l_A\sin(\alpha) \\
Y_E = -l_A\cos(\alpha),
\end{align}

the wrist joint point is

\begin{align}
X_W = X_E + l_F\sin(\alpha+\beta) \\
Y_W = Y_E - l_F\cos(\alpha+\beta)
\end{align}

and the coordinates for the fingertip point is

\begin{align}
X_H = X_W + l_H\sin(\alpha+\beta+\gamma) \\
Y_H = Y_W - l_H\cos(\alpha+\beta+\gamma) \label{lastcoord}
\end{align}

%The shape is determined from to the boundary conditions given in the %table\ref{tab:parameters}.

Table \ref{tab:parameters} shows typical values of the lengths of the arm, the forearm and the
hand and the limitations of the angles $\alpha$, $\beta$ and $\gamma$ for a person without any handicap.

\begin{table}[!h!]
\begin{center}
\begin{tabular}{|c|c|c|}
\hline
\bf Description & \bf Variable & \bf Constraints \\\hline
Arm length & $l_A$ &  28 cm \\\hline
Forearm length  & $l_F$ &  28 cm \\\hline
Palm length & $l_H$ &  18 cm  \\\hline
Shoulder angle & $\alpha$ & $[-50^{\circ},90^{\circ}]$ \\\hline
Elbow angle & $\beta$  & $[-10^{\circ},150^{\circ}]$  \\\hline
Hand angle & $\gamma$ & $[-70^{\circ},80^{\circ}]$  \\\hline
\end{tabular}
\caption{Standard parameters for an adult \cite{MR2345192}.}
\label{tab:parameters}
\end{center}
\end{table}

%It is possible to see in figure \ref{fig:shapebuild} the trajectory made by the fingertip point when %considering the standard parameters. %MOVE!
%
%\begin{figure}[!ht]
%  \centering
%  \includegraphics[width=0.45\textwidth]{figs/1.pdf}
%  \includegraphics[width=0.45\textwidth]{figs/200.pdf}
%  \includegraphics[width=0.45\textwidth]{figs/400.pdf}
%  \includegraphics[width=0.45\textwidth]{figs/500.pdf}
%  \includegraphics[width=0.45\textwidth]{figs/600.pdf}
%  \includegraphics[width=0.45\textwidth]{figs/700.pdf}
%  \includegraphics[width=0.45\textwidth]{figs/701.pdf}
%  \includegraphics[width=0.45\textwidth]{figs/800.pdf}
%\caption{Shape build process}
%\label{fig:shapebuild}
%\end{figure}

%Moreover, the area of the surface $\mathcal{A}$ generated by the hand motion is illustrated on %figure \ref{fig:boundary} for the standard parameters.

%\begin{figure}[!ht]
%  \centering
%  \includegraphics[scale=0.8]{figs/BoundaryOriginal}
%\caption{$\mathcal{A}$ with standard parameters}
%\label{fig:boundary}
%\end{figure}

\section{First step}

In order to get an idea about the geometry of the surface generated by the hand motion, the first thing done was to plot $(X_H, Y_H)$ for all the allowed values of $(\alpha, \beta, \gamma)$. The result is showed in figure \ref{fig:pointsOriginal}. The domain of all reachable points (blue in the figure) is denoted by $\mathcal{A}$.

\begin{figure}[!htp]
  \centering
  \includegraphics[width=0.7\linewidth]{figs/PointsOriginal}
\caption{$(X_H, Y_H)$ for all the allowed values of $(\alpha, \beta, \gamma)$. \label{fig:pointsOriginal}}
\label{fig:boundary}
\end{figure}

Then, analytic expressions for the different parts of the boundary were found. Figure \ref{fig:partsboundary} shows the boundary of the domain $\mathcal{A}$. The different parts are distinguished by distinct colors and shapes. For example, the dashed black line, $\tau_{H_4}$, comes from fixing $\alpha$ at $90^{\circ}$, $\gamma$ at $0^{\circ}$ and changing $\beta$ from $0^{\circ}$ to $150^{\circ}$. $\tau_{H_i}$, with $1 \leq i \leq 5$, stands for the several parts of the outer boundary and $\tau_{I_i}$, with $1 \leq i \leq 3$, stands for the several parts of the inner boundary. For other restrictions of $\alpha$, $\beta$ and $\gamma$ the boundary might have to be divided in a different way in order to find analytic expressions. %Explain?

\begin{figure}[!htp]
  \centering
  \includegraphics[width=0.8\linewidth]{figs/BoundaryOriginal}
\caption{Different parts of the boundary}
\label{fig:partsboundary}
\end{figure}

\section{Numerical Methods}

\subsection{The Monte-Carlo Method}

The Monte-Carlo method (MC) calculates the area $|\mathcal{A}|$ based on a random sampling algorithm.

%This algorithm was adapted to our specific case: equations \eqref{firstcoord} to \eqref{lastcoord}, which are based on $(\alpha,\beta,\gamma)$ and $(l_A, l_F, l_H)$, give only the coordinates of the fingertips. This means that we do not have an explicit way to decide whether a random point is inside or outside the domain $\mathcal{A}$. On figure \ref{fig:MCexp} there is an illustration of the MC decision process.

%\begin{figure}[!ht]
%  \centering
%  \label{fig:mcbound}
%  \includegraphics[scale=0.7]{figs/MCexplanationRap}
%  \caption{Illustration of the MC decision process}
% \label{fig:MCexp}
%\end{figure}

Before performing the MC-algorithm, a large number of points spread over the whole boundary of $\mathcal{A}$ has to be calculated. Then, denote the x-coordinate of the maximum point in the inner boundary as $x_r$ and its corresponding y-coordinate as $y_r$. Now, the MC simulation can be performed.

%First, we randomly distribute $N$ points in the smallest rectangle $\mathcal{A}_{rect}$ that %contains $\mathcal{A}$. To decide whether a point $P=(x_P,y_P)$ is inside or outside of %$\mathcal{A}$ we perform as follows:

Take a sample of $N$ random points uniformly distributed in the smallest rectangle $\mathcal{A}_{rect}$ that contains $\mathcal{A}$. To decide whether a point $P=(x_P,y_P)$ is inside or outside of $\mathcal{A}$ perform as follows:

\begin{enumerate}

\item Find all boundary points whose x-coordinates belong to $[x_P-\delta, x_P+\delta]$, where $\delta$ is a small number;

\item For the chosen boundary points, store its y-coordinates in a vector $\mathbf{Y_P}$;

\item If $x_P < x_r$ then:

    \begin{itemize}

    \item if $y_P < y_r$ then delete all values in $\mathbf{Y_P}$ that are larger than $y_r$;

    \item otherwise, delete all values in $\mathbf{Y_P}$ that are smaller than $y_r$;

    \end{itemize}

\item if $\min(\mathbf{Y_P}) \leq y_P \leq \max (\mathbf{Y_P})$ then $P \in \mathcal{A}$. Otherwise, $P \notin \mathcal{A}$.

\end{enumerate}

Figure \ref{fig:decision} gives an illustration of this procedure.

\begin{figure}[!ht]
  \centering
  \includegraphics[width=0.7\linewidth]{figs/MCexplanationRap}
  \caption{Illustration of the MC decision process. The green point is inside $\mathcal{A}$ and the red point is outside. The blue line marks the boundary of $\mathcal{A}$, the red lines mark $x_r$ and $y_r$ and the purple dashed lines show the interval $[x_P-\delta, x_P+\delta]$.  \label{fig:decision}}
 \label{fig:MCexp}
\end{figure}

In figure \ref{fig:MCOriginal}, we can see points randomly distributed in $\mathcal{A}_{rect}$. The ones that lay outside $\mathcal{A}$ are marked in red, the ones inside of $\mathcal{A}$ are marked in green. The points in the boundary of $\mathcal{A}$ are marked in blue.

\begin{figure}[!ht]
  \centering
  \label{fig:mcbound}
  \includegraphics[width=0.7\linewidth]{figs/MCOriginal}
  \caption{Set of points for the MC method}
 \label{fig:MCOriginal}
\end{figure}

In order to calculate $|\mathcal{A}|$, we count the number of green points $N_h$ and compute the area of $\mathcal{A}_{rect}$. Therefore,

\begin{equation}
|\mathcal{A}| \approx |\mathcal{A}_{rect}| \frac{N_h}{N} = \hat{|\mathcal{A}|}
\end{equation}

$\hat{|\mathcal{A}|}$ is a random variable that has expected value
\begin{equation}
\textnormal{E}\left[ \hat{|\mathcal{A}|}\right]=|\mathcal{A}|
\end{equation}
and the standard deviation
\begin{equation}\label{eq:sdMC}
\sigma_{\hat{|\mathcal{A}|}} =\sqrt{\frac{|\mathcal{A}|(|\mathcal{A}_{rect}| -|\mathcal{A}| )}{N}}
\end{equation}

\subsection{Green's Method}

Using Green's theorem \eqref{eq:green} is the second approach to obtain $|\mathcal{A}|$. This formula gives a relation between an integral over a simple closed curve, the boundary $\partial\mathcal{A}$ and the data on the whole surface of the domain $\mathcal{A}$. We will refer to this approach as \textit{Green'��s Method}.


\begin{equation}
 \int_{\partial \mathcal{A}} \frac{\partial p}{\partial \bf n} = \int_\mathcal{A} \Delta p
 \label{eq:green}
\end{equation}

The equation \eqref{eq:green} is equivalent to the following formulation

%
\begin{equation}
 \int_{\partial \mathcal{A}} \nabla p\cdot \mathbf{n} = \int_\mathcal{A} \Delta p
 \label{eq:green2}
\end{equation}
%
The unit vector of a curve is given by
\[
\mathbf n = \frac{1}{\sqrt{{\dot x}^2+{\dot y}^2}}
\begin{pmatrix}
 -\dot y \\
 \dot x
\end{pmatrix}
\]

Consider the several parts of the boundary visible in the figure \ref{fig:boundary}. The unit outward normal vectors are computed for the whole boundary. This vectors are needed to calculate the Green's theorem integrals.

For $\tau_{H_1}$, consisting of all the points fulfilling the following condition
\[
   \alpha=\alpha_{min},\;
   \beta=\beta_{min},\;
   \gamma_{min} < \gamma < 0,
\]
%
\[
\vec n_{H_1} =
\begin{pmatrix}
 -\sin(\alpha_{min} + \beta_{min} + \gamma) \\
  \cos(\alpha_{min} + \beta_{min} + \gamma)
\end{pmatrix}.
\]
%
For $\tau_{H_2}$, consisting of all the points fulfilling the following condition
\[
 \alpha=\alpha_{min},\;
 \beta_{min} < \beta < 0,\;
 \gamma=0,
\]
\[
\vec n_{H_2} =
\begin{pmatrix}
 -\sin(\alpha_{min} + \beta) \\
  \cos(\alpha_{min} + \beta)
\end{pmatrix}.
\]
%
For $\tau_{H_3}$, consisting of all the points fulfilling the following condition
\[
\alpha_{min} < \alpha < \alpha_{max}, \;
\beta=0, \;
\gamma=0,
\]
%
\[
\vec n_{H_3} =
\begin{pmatrix}
 -\sin(\alpha) \\
  \cos(\alpha)
\end{pmatrix}.
\]
%
For $\tau_{H_4}$, consisting of all the points fulfilling the following condition
\[
  \alpha=\alpha_{max},\;
  \beta_{min} < \beta < \beta_{max},\;
  \gamma = 0,
\]
%
\[
\vec n_{H_4} =
\begin{pmatrix}
 -\sin(\alpha_{max} + \beta) \\
  \cos(\alpha_{max} + \beta)
\end{pmatrix}.
\]
For $\tau_{H_5}$, consisting of all the points fulfilling the following condition
\[
  \alpha = \alpha_{max}, \;
  \beta = \beta_{max},\;
  0 < \gamma < \gamma_{max},
\]
%
\[
\vec n_{H_5} =
\begin{pmatrix}
 -\sin(\alpha_{max} + \beta_{max} + \gamma) \\
  \cos(\alpha_{max} + \beta_{max} + \gamma)
\end{pmatrix}.
\]
For $\tau_{I_1}$, consisting of all the points fulfilling the following condition
%
\[
 -\alpha_{min} < \alpha < \alpha_{max}, \;
 \beta = \beta_{max},\;
 \gamma=\gamma_{min},
\]
%
%
\[
\vec n_{I_1} =
\begin{pmatrix}
 -l_A \sin(\alpha) - l_F \sin(\alpha + \beta_{max}) - l_H\sin(\alpha+\beta_{max} + \gamma_{max}) \\
 l_A \cos(\alpha) + l_F \cos(\alpha + \beta_{max}) + l_H\cos(\alpha+\beta_{max} + \gamma_{max})
\end{pmatrix}.
\]
For $\tau_{I_2}$, consisting of all the points fulfilling the following condition
%
\[
  \alpha = \alpha_{min},\;
  \beta_{min} < \beta < \beta_{max},\;
  \gamma = \gamma_{min},
\]
%
%
\[
\vec n_{I_2} =
\begin{pmatrix}
 - l_F \sin(\alpha_{min} + \beta) - l_H\sin(\alpha_{min}+\beta + \gamma_{max}) \\
l_F \cos(\alpha_{min} + \beta) + l_H\cos(\alpha_{min}+\beta + \gamma_{max})
\end{pmatrix}.
\]
For $\tau_{I_3}$, consisting of all the points fulfilling the following condition
%
\[
  \alpha = \alpha_{min},\;
  \beta=\beta_{min},\;
  \gamma_{min} < \gamma < \gamma_{max},
\]
%
\[
\begin{pmatrix}
 -l_A \sin(\alpha) - l_F \sin(\alpha + \beta_{min}) - l_H\sin(\alpha+\beta_{min} + \gamma_{min}) \\
 l_A \cos(\alpha) + l_F \cos(\alpha + \beta_{min}) + l_H\cos(\alpha+\beta_{min} + \gamma_{min})
\end{pmatrix}.
\]
%
One has to be very careful in order to obtain the outward normal vectors. Sometimes a minus sign has to be added.
The figure \ref{fig:normv} illustrates these unit outward vectors along $\partial \mathcal{A}$.
%
\begin{figure}[!ht]
  \centering
  \label{fig:normv}
  \includegraphics[width=0.7\linewidth]{figs/norm}
  \caption{Unit outward normal vectors}
 \label{fig:normv}
\end{figure}

\noindent Choosing $p$ in such a way that $\Delta p=1$ in (\ref{eq:green2}) yields a formula for calculating $|\mathcal{A}|$.

Here, $p(x,y)=\frac{1}{2}x^2$ was chosen. This provided the (relatively easy) formula
\begin{equation}
 |\mathcal{A}| = \int_{\mathcal{A}} 1 = \int_\mathcal{A} \Delta p = \int_{\partial A} x\dot y
 \label{eq:green3}
\end{equation}
The last integral denotes a piece-wise integration over the different parts of the boundary curve (see figure \ref{fig:partsboundary}). Hence, an exact formulation for the desired area is obtained by determining these parts and summing up corresponding integrals.

The next section presents a comparison between both methods as well as some additional results.


\subsection{Results and Method Comparison}
Figure \ref{fig:compar} shows the reachable region for a person with full mobility, (a), and for three persons with reduced mobility (b), (c) and (d). Table \ref{tab:resultexampletable} shows the corresponding areas calculated with the two methods described in the previous section. The number of points used in the MC simulations was $N$=10 000.

\begin{figure}[!ht]
  \centering
  \subfigure[]{\includegraphics[width=0.49\linewidth]{figs/PointsOriginal}}
  \subfigure[]{\includegraphics[width=0.49\linewidth]{figs/Points2overlap}}
 \subfigure[]{\includegraphics[width=0.49\linewidth]{figs/Points3overlap}}
 \subfigure[]{\includegraphics[width=0.49\linewidth]{figs/Points4overlap}}
  \caption{Shape of the reachable area of the hand. \label{fig:compar}}
\end{figure}

\begin{table} 
\begin{center}
\begin{tabular}{|c|c|c|c|c|c|c|c|}
\hline
\bf Figure & $\alpha$ & \bf $\beta$ & \bf $\gamma$ & \bf MC area ($cm^2$)& \bf Green area $(cm^2$) \\\hline
(a) & [-50,90]  & [-10, 150] & [-70, 80] & 8011.2 (1e5) & 8064.6 \\\hline
(b)& [-20, 90] & [-10, 100]  & [-40, 70] & 4996.8 (1e5) & 5032.9\\\hline
(c) &[0,45] & [-10, 150] & [-70, 80] & 3530.2 (1e5) & 3339.2\\\hline
(d) & [0,50] & [25, 80] & [-20,60] & 1724.5 & 1724.5\\\hline
\end{tabular}
\end{center}
\caption{Examples of areas}
\label{tab:resultexampletable}
\end{table}

It is desirable to understand how the area $|\mathcal{A}|$ depends on the limitations of the angles. In order to do so, $|\mathcal{A}|$ is calculated while varying one angle limitation and keeping the others fixed at the values specified in \ref{tab:parameters}. Again, the calculations are done both with the method using Green's theorem and by MC-method with $N$=10 000 points. The results are presented in figures \ref{change_alpha}, \ref{change_beta} and \ref{change_gamma}. As can be seen, the area computed with MC-integration fluctuate around the area achieved with Green's theorem. This is due to the randomness inherent in MC-integration.

\begin{figure}[!ht]
  \centering
  \subfigure[]{\includegraphics[width=0.49\linewidth]{figs/alpha_min}}
  \subfigure[]{\includegraphics[width=0.49\linewidth]{figs/alpha_max}}
 \caption{Dependence between $|\mathcal{A}|$ and the limitations $\alpha$. \label{change_alpha}}
\end{figure}
\begin{figure}[!ht]
\subfigure[]{\includegraphics[width=0.49\linewidth]{figs/beta_min}}
\subfigure[]{\includegraphics[width=0.49\linewidth]{figs/beta_max}}
  \caption{Dependence between $|\mathcal{A}|$ and the limitations of $\beta$. \label{change_beta}}
\end{figure}
\begin{figure}[!ht]
  \subfigure[]{\includegraphics[width=0.49\linewidth]{figs/gamma_min}}
  \subfigure[]{\includegraphics[width=0.49\linewidth]{figs/gamma_max}}
  \caption{Dependence between $|\mathcal{A}|$ and the limitations of $\gamma$. \label{change_gamma}}
\end{figure}

Another subject of interest is to know how the accuracy and the computational time of the MC-integration depend on the number of points $N$. One measure of the accuracy is the standard deviation of $\hat{|\mathcal{A}|}$, which is given by equation \eqref{eq:sdMC}. Figure \ref{sdMC} shows the value of $\sigma_{\hat{|\mathcal{A}|}}$ for different values of $N$ with the limits of the angles given in table \ref{tab:parameters}.

\begin{figure}[!ht]
\centering
\includegraphics[width=0.7\linewidth]{figs/sdMC}
\caption{The standard deviation of $\hat{|\mathcal{A}|}$ for different values of $N$. \label{sdMC}}
\end{figure}

For the same angles, the area $|\mathcal{A}|$ was computed for different values of $N$. The results were compared with the value obtained using Green's theorem. Figure \ref{mc_green} shows the area computed with MC-integration divided by the area achieved using Green's theorem and figure \ref{mc_time} shows the time required for the MC-integration.

\begin{figure}[!ht]
\centering
\includegraphics[width=0.7\linewidth]{figs/mc_green}
\caption{$\hat{|\mathcal{A}|}/|\mathcal{A}|_{Green}$ for different values of $N$. \label{mc_green}}
\end{figure}

\begin{figure}[!ht]
\centering
\includegraphics[width=0.7\linewidth]{figs/mc_time}
\caption{Time required for MC-integration for different values of $N$.\label{mc_time} }
\end{figure}
%The set of points taken for computing the MC method are N=8.0689e3 (1e6) and  N=8.0256e3 (1e5).
%
%
%
%\begin{figure}[!ht]
%  \centering
%  \label{fig:compari}
%  \subfigure[]{\includegraphics[scale=0.5]{figs/PointsOriginal}}
%  \subfigure[]{\includegraphics[scale=0.5]{figs/Points2}}
%  \caption{Shape of the reachable area of the hand}
%\end{figure}
%
%\begin{table}
%\begin{center}
%\begin{tabular}{|c|c|c|c|c|c|c|}
%\hline
%\bf $\alpha$ & \bf $\beta$ & \bf $\gamma$ & \bf MC area ($cm^2$)& \bf Green area \\\hline
%[-50,90]  & [-10, 150] & [-70, 80] & 8.0112e3 (1e6) &\\\hline
%[-20, 90] & [-10, 100]  & [-40, 70] & 4.9968e3 (1e5) & \\\hline
%[0,45] & [-10, 150] & [-70, 80] & 3.5302e+003 (1e5) & \\\hline
%\end{tabular}
%\caption{Simulations}
%\label{tab:tests}
%\end{center}
%\end{table}
%
%Comparison pictures with different values of alpha



\section{Example of application}
In the following it is introduced a simple example of application. It simulates a real situation that a physiotherapist can be faced with.

Let us imagine that after an accident a patient can only slightly move the shoulder and the elbow, whereas the hand has not been affected:
$$
\alpha_0\in [-10^\circ, \; 10^\circ] \qquad \beta_0\in [0^\circ, \; 40^\circ] \qquad \gamma_0\in [-70^\circ, \; 80^\circ] \ .
$$

Figure \ref{fig:ex0a} shows the area that the patient can span. With MC-integration, $\mathcal{A}_0=1059.97$ cm$^2$.
\begin{figure}[!ht]
  \centering
  \includegraphics[width=0.7\linewidth]{figs/ex_0a}
\caption{Shape of the reachable area of the hand}
\label{fig:ex0a}
\end{figure}


We now assume that there are two possible treatments that the physiotherapist can follow:
\begin{enumerate}
\item This therapy focuses equally in improving the mobility of the two angles, and after one month both of them can be moved $40^\circ$ more:
$$
\alpha_1\in [-20^\circ, \; 40^\circ] \qquad \beta_1\in [0^\circ, \; 80^\circ] \qquad \gamma_1\in [-70^\circ, \; 80^\circ] \ .
$$

\item This therapy focuses more on the elbow, giving after one month an improvement of $60^\circ$ for this angle, and $20^\circ$ for the one corresponding to shoulder:
$$
\alpha_2\in [-10^\circ, \; 30^\circ] \qquad \beta_2\in [0^\circ, \; 100^\circ] \qquad \gamma_2\in [-70^\circ, \; 80^\circ] \ .
$$
\end{enumerate}

The shape of the possible movements for these two methods can be seen in Figure \ref{fig:ex1a}, whereas the values of the two areas ($A_1$ and $A_2$) are reported in Table \ref{tab:ex}.

\begin{figure}[!ht]
  \centering
  \includegraphics[width=0.49\linewidth]{figs/ex_1a}
  \hfill
  \includegraphics[width=0.49\linewidth]{figs/ex_2a}
\caption{Shape of the reachable area of the hand.}
\label{fig:ex1a}
\end{figure}

\begin{table}[!ht]
\begin{center}
\begin{tabular}{|c|c|c|}
\hline
\bf $\mathbf{\mathcal{A}_0}$ & \bf $\mathbf{\mathcal{A}_1}$ & \bf $\mathbf{\mathcal{A}_2}$ \\  \hline

1059.97 cm$^2$ & 3007.06 cm$^2$ &   2693.64 cm$^2$ \\\hline

\end{tabular}
\caption{Areas of the original mobility ($\mathcal{A}_0$) and of the two different treatments ($\mathcal{A}_1$ and $\mathcal{A}_2$)}
\label{tab:ex}
\end{center}
\end{table}

As we see from Table \ref{tab:ex}, the first treatment allows the patient to reach a bigger area of points. However, some of the improvement is for points behind the patient (notice that $\alpha_{\text{min}}$ improves from -10$^\circ$ to -20$^\circ$) that are for sure less important than points in front of the patient.

It is therefore interesting to introduce weights that give more importance to points in a certain region. Figure \ref{fig:w} shows an example of a Gaussian weighting, truncated such that points behind the patient have no importance, and with peak in front of the patient's chest.

\begin{figure}[!ht]
  \centering
  \includegraphics[width=0.7\linewidth]{figs/weights}
\caption{Gaussian weighting of points in space}
\label{fig:w}
\end{figure}

If we are using the Monte-Carlo Integration, the weighting is straightforward, as we have $N$ samples that belong to a uniform distribution over the area spanned by the fingertips.
In this case we can in fact weight each point using the value of the Gaussian distribution in that position, and summing the contribution of all the samples obtaining a quantity that we call \textit{Importance}:
$$
I=\sum_{(x,y)\in A} f(x,y) \ ,
$$
where $f(x,y)$ is the 2d Gaussian distribution. If we define $\mathbf{p}=(x,y)$, $\text{\boldmath $\mu$}$ the mean vector and $\mathbf{\Sigma}$ the $2\times 2$ covariance matrix, we have
$$
f(\mathbf{p})=
\begin{cases}
Ce^{-\frac{1}{2}(\mathbf{p}-\text{\boldmath $\mu$})^T\mathbf{\Sigma}^{-1}(\mathbf{p}-\text{\boldmath $\mu$})} \qquad & \text{if } x>20\text{ cm} \\
0 & \text{otherwise}
\end{cases}
$$





The Gaussian weighting and the simulated points for the two possible treatments are shown in Figure \ref{fig:ex1b}.

\begin{figure}[!ht]
  \centering
  \includegraphics[width=0.49\linewidth]{figs/ex_1b}
  \hfill
  \includegraphics[width=0.49\linewidth]{figs/ex_2b}
\caption{MC samples and weights.}
\label{fig:ex1b}
\end{figure}

If we evaluate the Importance for the three configurations we get the results given in Table \ref{tab:exI}.

\begin{table}[!ht]
\begin{center}
\begin{tabular}{|c|c|c|}
\hline
\bf $\mathbf{I_0}$ & \bf $\mathbf{I_1}$ & \bf $\mathbf{I_2}$ \\  \hline

 0.081 &   0.1806 &   0.1968 \\\hline

\end{tabular}
\caption{Importances with respect to $\mathcal{A}_0$,$\mathcal{A}_1$ and $\mathcal{A}_2$}
\label{tab:exI}
\end{center}
\end{table}

It is interesting to notice that even if the area previously returned by treatment 1 is higher (see Table \ref{tab:ex}), the information provided by the Importance tells us that to maximize the weighted area treatment 2 is better.
From Figure \ref{fig:comp} we can in fact see that for treatment 2 we consider more points in the peak of the Gaussian than for treatment 1, and that the latter takes lots of points in the region with zero weight (these samples contribute to the evaluation of the area but not that of the importance).

\begin{figure}[!ht]
  \centering
  \includegraphics[width=0.7\textwidth]{figs/comparison}
  \caption{MC samples and weights.}
\label{fig:comp}
\end{figure}



\section{Conclusions}

%For doctors the most important is to observe the progress in
%treatment for disabled people and if it is needed locate it to the correct
%direction.
%
%For people who are involved in professional sports and has arm
%diseases, it is very important to find problem and needed treatment as soon as
%it can be possible.
%
%For a group of healthy people with ages from 18 till 40, angles and length of joint parts of the arm were measured. Based on such information, the model of the human-arm movement was built.
%
%Thus this
%project is dedicated to calculation the percentage of the reachable area for
%human arm deasis based on the same information for healthy people.
%
%For
%calculation this area it was used 2 numerical methods: Monte-Carlo method and
%method based on Green's theorem. Main problem in implementation for both
%methods was to creat peace wise function for boundaries of reachable area.
%
%For
%Monte-Carlo method this problem was decided based on knowledge of coordinates
%for every point of the boundary and for numerical method and for Green's
%theorem it was done based on on properties Afterwards, achieved results
%after methods implementation have the same quality with small
%computational error and can be used separately or in the cooperation
%for checking results.  Thus implemented model for calculation the
%percentage of human arm reachable area can be used in treatment
%for arm disease for seeing progress after physical
%rehabilitation.
This report shows two different methods to calculate the area of the surface generated by the hand motion. 
The major difficulty faced was to obtain the analytic expressions for the several parts of the boundary of $\mathcal{A}$. Our approach performs well for the majority of cases but it was not found the optimal solution.

Concerning future work, there are many improvements that can be done on our model. For instance, it would be important to consider the movement of the fingers. On the other hand, simulations regarding frontal and transversal planes could also be considered.

Work with a specialist, for instance a physiotherapist, would help to develop practical applications.






